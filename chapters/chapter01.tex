\chapter{Zagadnienia wstępne}
\section{Definicje}

\dfn{Funkcja Collatza}{
Funkcją Collatza nazywamy funkcję 
\begin{equation*}
f(n)=
    \begin{cases}
        \frac{n}{2} & \text{gdy } n \in \mathbb{N}_{PAR} \\
        3n + 1 & \text{gdy } n \in \mathbb{N}_{NPAR}
    \end{cases}
\end{equation*}
}

\thm{}{If $x\in$ open set $V$ then $\exists$ $\delta>0$ such that $B_{\delta}(x)\subset V$}

\ex{}{To jest przykład}



\qs{Problem Collatza}{Pytanie, czy problem Collatza $3n + 1$ ma rozwiązanie}
\sol Czasem tak, czasem nie



\nt{Moja notka moja notka}

\cor{Zatem zawsze 42.}


% code from http://rosettacode.org/wiki/Fibonacci_sequence#Python
\begin{lstlisting}[label={list:first},caption=Sample Python code -- Fibonacci sequence calculated analytically.]
from math import *

# define function 
def analytic_fibonacci(n):
  sqrt_5 = sqrt(5);
  p = (1 + sqrt_5) / 2;
  q = 1/p;
  return int( (p**n + q**n) / sqrt_5 + 0.5 )
 
# define range
for i in range(1,31):
  print analytic_fibonacci(i)
\end{lstlisting}

Following Listing~\ref{list:first}\ldots{} 
Lorem ipsum dolor sit amet, consectetur adipiscing elit, sed do eiusmod tempor incididunt ut labore et dolore magna aliqua. Ut enim ad minim veniam, quis nostrud exercitation ullamco laboris nisi ut aliquip ex ea commodo consequat. Duis aute irure dolor in reprehenderit in voluptate velit esse cillum dolore eu fugiat nulla pariatur. Excepteur sint occaecat cupidatat non proident, sunt in culpa qui officia deserunt mollit anim id est laborum.

\section*{Problem 2}

\begin{lstlisting}[label={list:second},caption=Sample Bash code.]
#! /bin/bash
python stage1.py
echo "Stage I done!"
python stage2.py
echo "Stage II done!"
python stage3.py
echo "Stage III done!"
\end{lstlisting}

