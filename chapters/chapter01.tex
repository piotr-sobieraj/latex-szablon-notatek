\chapter{Zagadnienia wstępne}
\section{Definicje}

\dfn{Funkcja Collatza}{
Funkcją Collatza nazywamy funkcję 
\begin{equation*}
f(n)=
    \begin{cases}
        \frac{n}{2} & \text{gdy } n \in \mathbb{N}_{PAR} \\
        3n + 1 & \text{gdy } n \in \mathbb{N}_{NPAR}
    \end{cases}
\end{equation*}
}

\thm{}{If $x\in$ open set $V$ then $\exists$ $\delta>0$ such that $B_{\delta}(x)\subset V$}

\ex{}{To jest przykład}



\qs{Problem Collatza}{Pytanie, czy problem Collatza $3n + 1$ ma rozwiązanie}
\sol Czasem tak, czasem nie



\nt{Moja notka moja notka}

