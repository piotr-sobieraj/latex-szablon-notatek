\documentclass[12pt]{report}

\input{preamble}
\input{macros}
\input{letterfonts}


\title{\Huge{Szablon notatek}}
\author{\huge{Piotr Sobieraj}}
\date{wiosna 2023}

\begin{document}

\maketitle
\newpage
\pdfbookmark[section]{\contentsname}{toc}
\tableofcontents


\chapter{}
\section{Zagadnienia wstępne}

\qs{Problem Collatza}{Pytanie, czy problem Collatza $3n + 1$ ma rozwiązanie}

\dfn{Funkcja Collatza}{
Funkcją Collatza nazywamy funkcję 
\begin{equation*}
f(n)=
    \begin{cases}
        \frac{n}{2} & \text{gdy } n \in \mathbb{N}_{PAR} \\
        3n + 1 & \text{gdy } n \in \mathbb{N}_{NPAR}
    \end{cases}
\end{equation*}
}

\ex{}




\end{document}
